% Options for packages loaded elsewhere
\PassOptionsToPackage{unicode}{hyperref}
\PassOptionsToPackage{hyphens}{url}
%
\documentclass[
]{book}
\usepackage{amsmath,amssymb}
\usepackage{lmodern}
\usepackage{iftex}
\ifPDFTeX
  \usepackage[T1]{fontenc}
  \usepackage[utf8]{inputenc}
  \usepackage{textcomp} % provide euro and other symbols
\else % if luatex or xetex
  \usepackage{unicode-math}
  \defaultfontfeatures{Scale=MatchLowercase}
  \defaultfontfeatures[\rmfamily]{Ligatures=TeX,Scale=1}
\fi
% Use upquote if available, for straight quotes in verbatim environments
\IfFileExists{upquote.sty}{\usepackage{upquote}}{}
\IfFileExists{microtype.sty}{% use microtype if available
  \usepackage[]{microtype}
  \UseMicrotypeSet[protrusion]{basicmath} % disable protrusion for tt fonts
}{}
\makeatletter
\@ifundefined{KOMAClassName}{% if non-KOMA class
  \IfFileExists{parskip.sty}{%
    \usepackage{parskip}
  }{% else
    \setlength{\parindent}{0pt}
    \setlength{\parskip}{6pt plus 2pt minus 1pt}}
}{% if KOMA class
  \KOMAoptions{parskip=half}}
\makeatother
\usepackage{xcolor}
\IfFileExists{xurl.sty}{\usepackage{xurl}}{} % add URL line breaks if available
\IfFileExists{bookmark.sty}{\usepackage{bookmark}}{\usepackage{hyperref}}
\hypersetup{
  pdftitle={财政学},
  pdfauthor={谭皓文},
  hidelinks,
  pdfcreator={LaTeX via pandoc}}
\urlstyle{same} % disable monospaced font for URLs
\usepackage{longtable,booktabs,array}
\usepackage{calc} % for calculating minipage widths
% Correct order of tables after \paragraph or \subparagraph
\usepackage{etoolbox}
\makeatletter
\patchcmd\longtable{\par}{\if@noskipsec\mbox{}\fi\par}{}{}
\makeatother
% Allow footnotes in longtable head/foot
\IfFileExists{footnotehyper.sty}{\usepackage{footnotehyper}}{\usepackage{footnote}}
\makesavenoteenv{longtable}
\usepackage{graphicx}
\makeatletter
\def\maxwidth{\ifdim\Gin@nat@width>\linewidth\linewidth\else\Gin@nat@width\fi}
\def\maxheight{\ifdim\Gin@nat@height>\textheight\textheight\else\Gin@nat@height\fi}
\makeatother
% Scale images if necessary, so that they will not overflow the page
% margins by default, and it is still possible to overwrite the defaults
% using explicit options in \includegraphics[width, height, ...]{}
\setkeys{Gin}{width=\maxwidth,height=\maxheight,keepaspectratio}
% Set default figure placement to htbp
\makeatletter
\def\fps@figure{htbp}
\makeatother
\setlength{\emergencystretch}{3em} % prevent overfull lines
\providecommand{\tightlist}{%
  \setlength{\itemsep}{0pt}\setlength{\parskip}{0pt}}
\setcounter{secnumdepth}{5}
\usepackage{ctex}

%\usepackage{xltxtra} % XeLaTeX的一些额外符号
% 设置中文字体
%\setCJKmainfont[BoldFont={黑体},ItalicFont={楷体}]{新宋体}

% 设置边距
\usepackage{geometry}
\geometry{%
  left=2.0cm, right=2.0cm, top=3.5cm, bottom=2.5cm} 

\usepackage{amsthm,mathrsfs}
\usepackage{booktabs}
\usepackage{longtable}
\makeatletter
\def\thm@space@setup{%
  \thm@preskip=8pt plus 2pt minus 4pt
  \thm@postskip=\thm@preskip
}
\makeatother
\ifLuaTeX
  \usepackage{selnolig}  % disable illegal ligatures
\fi
\usepackage[style=apa,]{biblatex}
\addbibresource{mybib.bib}

\title{财政学}
\author{谭皓文}
\date{2021年8月31日}

\usepackage{amsthm}
\newtheorem{theorem}{定理}[chapter]
\newtheorem{lemma}{Lemma}[chapter]
\newtheorem{corollary}{Corollary}[chapter]
\newtheorem{proposition}{Proposition}[chapter]
\newtheorem{conjecture}{Conjecture}[chapter]
\theoremstyle{definition}
\newtheorem{definition}{定义}[chapter]
\theoremstyle{definition}
\newtheorem{example}{例}[chapter]
\theoremstyle{definition}
\newtheorem{exercise}{Exercise}[chapter]
\theoremstyle{definition}
\newtheorem{hypothesis}{Hypothesis}[chapter]
\theoremstyle{remark}
\newtheorem*{remark}{Remark}
\newtheorem*{solution}{解: }
\begin{document}
\maketitle

{
\setcounter{tocdepth}{1}
\tableofcontents
}
\hypertarget{ux5bfcux8bba}{%
\chapter{导论}\label{ux5bfcux8bba}}

\hypertarget{ux516cux6c11ux4e0eux8d22ux653fux4e4bux95f4ux7684ux5173ux7cfb}{%
\section{公民与财政之间的关系}\label{ux516cux6c11ux4e0eux8d22ux653fux4e4bux95f4ux7684ux5173ux7cfb}}

\hypertarget{ux8d22ux653fux5bf9ux4e8eux516cux6c11ux4e2aux4f53ux7684ux4fddux969c}{%
\subsection{财政对于公民个体的保障}\label{ux8d22ux653fux5bf9ux4e8eux516cux6c11ux4e2aux4f53ux7684ux4fddux969c}}

这些保障体现为

\begin{itemize}
\tightlist
\item
  医疗保障
\item
  教育保障
\item
  充分就业保障、失业保障
\item
  养老保障
\item
  \ldots\ldots{}
\end{itemize}

\hypertarget{ux8d22ux653fux5bf9ux4e8eux516cux6c11ux6574ux4f53ux7684ux4fddux969c}{%
\subsection{财政对于公民整体的保障}\label{ux8d22ux653fux5bf9ux4e8eux516cux6c11ux6574ux4f53ux7684ux4fddux969c}}

\begin{itemize}
\tightlist
\item
  国防
\item
  公共安全
\item
  行政管理
\item
  基础设施
\item
  \ldots\ldots{}
\end{itemize}

其中,国防与公共安全属于纯公共产品,\textbf{其特点:不直接付费},而基础设施这些例如公共交通等,需要付费使用。

\hypertarget{ux516cux6c11ux5bf9ux4e8eux8d22ux653fux7684ux8d21ux732e}{%
\subsection{公民对于财政的贡献}\label{ux516cux6c11ux5bf9ux4e8eux8d22ux653fux7684ux8d21ux732e}}

\begin{itemize}
\tightlist
\item
  依法纳税
\item
  购买国债
\end{itemize}

\hypertarget{ux793eux4f1aux516cux5171ux9700ux8981ux4e0eux516cux5171ux4ea7ux54c1}{%
\section{社会公共需要与公共产品}\label{ux793eux4f1aux516cux5171ux9700ux8981ux4e0eux516cux5171ux4ea7ux54c1}}

\hypertarget{ux793eux4f1aux516cux5171ux9700ux8981}{%
\subsection{社会公共需要}\label{ux793eux4f1aux516cux5171ux9700ux8981}}

\textbf{社会公共需要含义}:区别于私人个别需要,由公共部门提供的,满足社会整体的需求

\textbf{社会公共需要的特征}:

\begin{itemize}
\tightlist
\item
  社会公共的共同特征
\item
  均等性与福利性
\item
  与经济发展相适应
\end{itemize}

\hypertarget{ux516cux5171ux4ea7ux54c1}{%
\subsection{公共产品}\label{ux516cux5171ux4ea7ux54c1}}

\textbf{公共产品定义}:由公共部门提供的,满足社会公共需要的产品

\textbf{公共产品的特征}:

\begin{itemize}
\tightlist
\item
  非排他性
\item
  非竞争性
\end{itemize}

\begin{quote}
部分准公共产品具有竞争性
\end{quote}

\hypertarget{ux8d22ux653fux653fux7b56}{%
\subsection{财政政策}\label{ux8d22ux653fux653fux7b56}}

\textbf{财政的概念}: 政府统一进行资金管理,实现国家经济社会发展职能的分配行为。

\textbf{财政的研究对象和内容}:

\begin{itemize}
\tightlist
\item
  研究对象: 社会主义市场经济体制下的公共财政分配活动和分配关系及其规律性
\item
  内容: 基本理论,财政支出,财政收入,财政管理与政策
\end{itemize}

\hypertarget{ux8d22ux653fux4e0eux5176ux4ed6ux5b66ux79d1ux4e4bux95f4ux7684ux5173ux7cfb}{%
\subsection{财政与其他学科之间的关系}\label{ux8d22ux653fux4e0eux5176ux4ed6ux5b66ux79d1ux4e4bux95f4ux7684ux5173ux7cfb}}

\ldots\ldots\ldots\ldots\ldots\ldots{}

\hypertarget{ux5f53ux4ee3ux8d22ux653fux70edux70b9ux95eeux9898}{%
\subsection{当代财政热点问题}\label{ux5f53ux4ee3ux8d22ux653fux70edux70b9ux95eeux9898}}

\begin{quote}
可作为小组展示的话题选题
\end{quote}

\begin{itemize}
\tightlist
\item
  地方政府债务风险问题
\item
  土地财政问题
\item
  分税制改革问题
\item
  延迟退休问题
\item
  财政政策与货币政策协调问题
\item
  收入分配问题
\item
  供给侧结构性改革
\item
  PPP与投融资体质改革
\end{itemize}

\hypertarget{ux804cux80fd}{%
\chapter{财政的职能}\label{ux804cux80fd}}

\hypertarget{ux8d22ux653fux804cux80fdux6982ux8ff0}{%
\section{财政职能概述}\label{ux8d22ux653fux804cux80fdux6982ux8ff0}}

\hypertarget{ux8d22ux653fux7684ux8d44ux6e90ux914dux7f6eux804cux80fd}{%
\section{财政的资源配置职能}\label{ux8d22ux653fux7684ux8d44ux6e90ux914dux7f6eux804cux80fd}}

\hypertarget{ux8d44ux6e90ux914dux7f6eux7684ux542bux4e49}{%
\subsection{资源配置的含义}\label{ux8d44ux6e90ux914dux7f6eux7684ux542bux4e49}}

\hypertarget{ux8d44ux6e90ux914dux7f6eux7684ux65b9ux5f0f}{%
\subsection{资源配置的方式}\label{ux8d44ux6e90ux914dux7f6eux7684ux65b9ux5f0f}}

\hypertarget{ux8d22ux653fux7684ux6536ux5165ux5206ux914dux804cux80fd}{%
\section{财政的收入分配职能}\label{ux8d22ux653fux7684ux6536ux5165ux5206ux914dux804cux80fd}}

\hypertarget{ux6536ux5165ux5206ux914dux7684ux6db5ux4e49}{%
\subsection{收入分配的涵义}\label{ux6536ux5165ux5206ux914dux7684ux6db5ux4e49}}

\begin{itemize}
\tightlist
\item
  初次分配
\item
  再分配
\item
  三次分配
\end{itemize}

\begin{quote}
三次分配的含义
\end{quote}

\hypertarget{ux8d22ux653fux7684ux6536ux5165ux5206ux914dux529fux80fd}{%
\subsection{财政的收入分配功能}\label{ux8d22ux653fux7684ux6536ux5165ux5206ux914dux529fux80fd}}

\hypertarget{ux8d22ux653fux8c03ux8282ux6536ux5165ux7684ux65b9ux5f0fux4e0eux624bux6bb5}{%
\subsection{财政调节收入的方式与手段}\label{ux8d22ux653fux8c03ux8282ux6536ux5165ux7684ux65b9ux5f0fux4e0eux624bux6bb5}}

\begin{itemize}
\tightlist
\item
  税收
\item
  转移支付
\end{itemize}

\hypertarget{ux8d22ux653fux7684ux7ecfux6d4eux7a33ux5b9aux4e0eux53d1ux5c55ux804cux80fd}{%
\section{财政的经济稳定与发展职能}\label{ux8d22ux653fux7684ux7ecfux6d4eux7a33ux5b9aux4e0eux53d1ux5c55ux804cux80fd}}

\hypertarget{ux5b8fux89c2ux7ecfux6d4eux8c03ux63a7ux7684ux56dbux5927ux76eeux6807}{%
\subsection{宏观经济调控的四大目标}\label{ux5b8fux89c2ux7ecfux6d4eux8c03ux63a7ux7684ux56dbux5927ux76eeux6807}}

\begin{itemize}
\tightlist
\item
  经济增长
\item
  物价稳定
\item
  充分就业
\item
  国际收支平衡
\end{itemize}

\hypertarget{ux8d22ux653fux5728ux5b9eux73b0ux56dbux5927ux76eeux6807ux4e2dux7684ux4f5cux7528}{%
\subsection{财政在实现四大目标中的作用}\label{ux8d22ux653fux5728ux5b9eux73b0ux56dbux5927ux76eeux6807ux4e2dux7684ux4f5cux7528}}

\hypertarget{ux8d22ux653fux804cux80fdux4e0eux516cux5e73ux6548ux7387ux51c6ux5219}{%
\section{财政职能与公平效率准则}\label{ux8d22ux653fux804cux80fdux4e0eux516cux5e73ux6548ux7387ux51c6ux5219}}

从效率优先兼顾公平到效率优先与公平并重

\hypertarget{causal}{%
\chapter{格兰格因果性}\label{causal}}

\hypertarget{causal-intro}{%
\section{介绍}\label{causal-intro}}

考虑两个时间序列之间的因果性。
这里的因果性指的是时间顺序上的关系,
如果\(X_{t-1}, X_{t-2}, \dots\)对\(Y_t\)有作用,
而\(Y_{t-1}, Y_{t-2}, \dots\)对\(X_t\)没有作用,
则称\(\{X_t \}\)是\(\{ Y_t \}\)的格兰格原因,
而\(\{ Y_t \}\)不是\(\{ X_t \}\)的格兰格原因。
如果\(X_{t-1}, X_{t-2}, \dots\)对\(Y_t\)有作用,
\(Y_{t-1}, Y_{t-2}, \dots\)对\(X_t\)也有作用,
则在没有进一步信息的情况下无法确定两个时间序列的因果性关系。

注意这种因果性与采样频率有关系,
在日数据或者月度数据中能发现的领先------滞后性质的因果关系,
到年度数据可能就以及混杂在以前变成同步的关系了。

\hypertarget{causal-def}{%
\section{格兰格因果性的定义}\label{causal-def}}

设\(\{ \xi_t \}\)为一个时间序列,
\(\{ \boldsymbol{\eta}_t \}\)为向量时间序列,
记
\[\begin{aligned}
\bar{\boldsymbol{\eta}}_t =& \{ \boldsymbol{\eta}_{t-1}, \boldsymbol{\eta}_{t-2}, \dots \} 
\end{aligned}\]

记
\(\text{Pred}(\xi_t | \bar{\boldsymbol{\eta}}_t)\)为基于
\(\boldsymbol{\eta}_{t-1}, \boldsymbol{\eta}_{t-2}, \dots\)
对\(\xi_t\)作的最小均方误差无偏预报,
其解为条件数学期望\(E(\xi_t | \boldsymbol{\eta}_{t-1}, \boldsymbol{\eta}_{t-2}, \dots)\),
在一定条件下可以等于\(\xi_t\)在\(\boldsymbol{\eta}_{t-1}, \boldsymbol{\eta}_{t-2}, \dots\)张成的线性Hilbert空间的投影
(比如,\((\xi_t, \boldsymbol{\eta}_t)\)为平稳正态多元时间序列),
即最优线性预测。
直观理解成基于过去的\(\{\boldsymbol{\eta}_{t-1}, \boldsymbol{\eta}_{t-2}, \dots \}\)的信息对当前的\(\xi_t\)作的最优预测。

令一步预测误差为
\[
  \varepsilon(\xi_t | \bar{\boldsymbol{\eta}}_t) 
  = \xi_t - \text{Pred}(\xi_t | \bar{\boldsymbol{\eta}}_t)
\]
令一步预测误差方差,或者均方误差,
为
\[
  \sigma^2(\xi_t | \bar{\boldsymbol{\eta}}_t)  
  = \text{Var}(\varepsilon_t(\xi_t | \bar{\boldsymbol{\eta}}_t) )
  = E \left[ \xi_t - \text{Pred}(\xi_t | \bar{\boldsymbol{\eta}}_t) \right]^2
\]

考虑两个时间序列\(\{ X_t \}\)和\(\{ Y_t \}\),
\(\{(X_t, Y_t) \}\)宽平稳或严平稳。

\begin{itemize}
\tightlist
\item
  如果
  \[
  \sigma^2(Y_t | \bar Y_t, \bar X_t) < \sigma^2(Y_t | \bar Y_t)
  \]
  则称\(\{ X_t \}\)是\(\{ Y_t \}\)的\textbf{格兰格原因},
  记作\(X_t \Rightarrow Y_t\)。
  这不排除\(\{ Y_t \}\)也可以是\(\{ X_t \}\)的格兰格原因。
\item
  如果\(X_t \Rightarrow Y_t\),而且\(Y_t \Rightarrow X_t\),
  则称互相有\textbf{反馈}关系,
  记作\(X_t \Leftrightarrow Y_t\)。
\item
  如果
  \[
  \sigma^2(Y_t | \bar Y_t, X_t, \bar X_t) < \sigma^2(Y_t | \bar Y_t, \bar X_t)
  \]
  即除了过去的信息,
  增加同时刻的\(X_t\)信息后对\(Y_t\)预测有改进,
  则称\(\{X_t \}\)对\(\{Y_t \}\)有瞬时因果性。
  这时\(\{Y_t \}\)对\(\{X_t \}\)也有瞬时因果性。
\item
  如果\(X_t \Rightarrow Y_t\),
  则存在最小的正整数\(m\),
  使得
  \[
  \sigma^2(Y_t | \bar Y_t, X_{t-m}, X_{t-m-1}, \dots) 
  < \sigma^2(Y_t | \bar Y_t, X_{t-m-1}, X_{t-m-2}, \dots) 
  \]
  称\(m\)为\textbf{因果性滞后值}(causality lag)。
  如果\(m>1\),
  这意味着在已有\(Y_{t-1}, Y_{t-2}, \dots\)和\(X_{t-m}, X_{t-m-1}, \dots\)的条件下,
  增加\(X_{t-1}\), \dots, \(X_{t-m+1}\)不能改进对\(Y_t\)的预测。
\end{itemize}

\begin{example}
\protect\hypertarget{exm:causal-exaxylag1}{}{\label{exm:causal-exaxylag1} }设\(\{ \varepsilon_t, \eta_t \}\)是相互独立的零均值白噪声列,
\(\text{Var}(\varepsilon_t)=1\),
\(\text{Var}(\eta_t)=1\),
考虑
\[\begin{aligned}
Y_t =& X_{t-1} + \varepsilon_t \\
X_t =& \eta_t + 0.5 \eta_{t-1}
\end{aligned}\]
\end{example}

用\(L(\cdot|\cdot)\)表示最优线性预测,则
\[\begin{aligned}
& L(Y_t | \bar Y_t, \bar X_t) \\
=& L(X_{t-1} | X_{t-1}, \dots, Y_{t-1}, \dots)
+ L(\varepsilon_t | \bar Y_t, \bar X_t) \\
=& X_{t-1} + 0 \\
=& X_{t-1} \\
\sigma(Y_t | \bar Y_t, \bar X_t) =&
\text{Var}(\varepsilon_t) = 1
\end{aligned}\]
而
\[
Y_t = \eta_{t-1} + 0.5\eta_{t-2} + \varepsilon_t
\]
有
\[\begin{aligned}
\gamma_Y(0) = 2.25,
\gamma_Y(1) = 0.5,
\gamma_Y(k) = 0, k \geq 2
\end{aligned}\]
所以\(\{Y_t \}\)是一个MA(1)序列,
设其方程为
\[
Y_t = \zeta_t + b \zeta_{t-1}, 
\zeta_t \sim \text{WN}(0, \sigma_\zeta^2)
\]
可以解出
\[\begin{aligned}
\rho_Y(1) =& \frac{\gamma_Y(1)}{\gamma_Y(0)} = \frac{2}{9} \\
b =& \frac{1 - \sqrt{1 - 4 \rho_Y^2(1)}}{2 \rho_Y(1)}
\approx 0.2344 \\
\sigma_\zeta^2 =& \frac{\gamma_Y(1)}{b} \approx 2.1328
\end{aligned}\]
于是
\[\begin{aligned}
\sigma(Y_t | \bar Y_t)
=& \sigma_\zeta^2 = 2.1328
\end{aligned}\]
所以
\[\begin{aligned}
\sigma(Y_t | \bar Y_t, \bar X_t) = 1
< 2.1328 = \sigma(Y_t | \bar Y_t)
\end{aligned}\]
即\(X_t\)是\(Y_t\)的格兰格原因。

反之,
\(X_t\)是MA(1)序列,
有
\[
\eta_t = \frac{1}{1 + 0.5 B} X_t
= \sum_{j=0}^\infty (-0.5)^j X_{t-j}
\]
其中\(B\)是推移算子(滞后算子)。
于是
\[\begin{aligned}
L(X_t | \bar X_t)
=& L(\eta_t | \bar X_t)
+ 0.5 L(\eta_{t-1} | \bar X_t) \\
=& 0.5 \sum_{j=0}^\infty (-0.5)^j X_{t-1-j} \\
=& - \sum_{j=1}^\infty (-0.5)^j X_{t-j} \\
\sigma(X_t | \bar X_t)
=& \text{Var}(X_t - L(X_t | \bar X_t)) \\
=& \text{Var}(\eta_t) = 1
\end{aligned}\]
而
\[\begin{aligned}
L(X_t | \bar X_t, \bar Y_t) 
=& L(\eta_t | \bar X_t, \bar Y_t)
+ 0.5 L(\eta_{t-1} | \bar X_t, \bar Y_t) \\
=& 0 +
0.5 L(\sum_{j=0}^\infty (-0.5)^j X_{t-1-j} | \bar X_t, \bar Y_t) \\
=& -\sum_{j=1}^\infty (-0.5)^j X_{t-j} \\
=& L(X_t | \bar X_t)
\end{aligned}\]
所以\(Y_t\)不是\(X_t\)的格兰格原因。

考虑瞬时因果性。
\[\begin{aligned}
L(Y_t | \bar X_t, \bar Y_t, X_t)
=& X_{t-1} + 0 (\text{注意}\varepsilon_t\text{与}\{X_s, \forall s\}\text{不相关} \\
=& L(Y_t | \bar X_t, \bar Y_t)
\end{aligned}\]
所以\(X_t\)不是\(Y_t\)的瞬时格兰格原因。

○○○○○

\begin{example}
\protect\hypertarget{exm:causal-exaxylag2}{}{\label{exm:causal-exaxylag2} }在例\ref{exm:causal-exaxylag1}中,如果模型改成
\[\begin{aligned}
Y_t =& X_{t} + \varepsilon_t \\
X_t =& \eta_t + 0.5 \eta_{t-1}
\end{aligned}\]
有怎样的结果?
\end{example}

这时
\[
Y_t = \varepsilon_t + \eta_t + 0.5 \eta_{t-1}
\]
仍有
\[\begin{aligned}
\gamma_Y(0) = 2.25,
\gamma_Y(1) = 0.5,
\gamma_Y(k) = 0, k \geq 2
\end{aligned}\]
所以\(Y_t\)还服从MA(1)模型
\[
Y_t = \zeta_t + b \zeta_{t-1},
b \approx 0.2344,
\sigma^2_\zeta \approx 2.1328
\]

\[\begin{aligned}
L(Y_t | \bar Y_t, \bar X_t)
=& L(X_t | \bar Y_t, \bar X_t) + 0 \\
=& L(\eta_t | \bar Y_t, \bar X_t)
+ 0.5 L(\eta_{t-1} | \bar Y_t, \bar X_t) \\
=& 0 + 0.5 L(\sum_{j=0}^\infty (-0.5)^j X_{t-1-j} | \bar Y_t, \bar X_t) \\
=& - \sum_{j=1}^\infty (-0.5)^j X_{t-j} \\
=& X_t - \eta_t \\
\sigma(Y_t | \bar Y_t, \bar X_t) 
=& \text{Var}(\varepsilon_t + \eta_t) = 2
\end{aligned}\]
而
\[
\sigma(Y_t | \bar Y_t)
= \sigma^2_\zeta \approx 2.1328
> \sigma(Y_t | \bar Y_t, \bar X_t) = 2
\]
所以\(X_t\)是\(Y_t\)的格兰格原因。

反之,
\[\begin{aligned}
L(X_t | \bar X_t, \bar Y_t)
=& - \sum_{j=1}^\infty (-0.5)^j X_{t-j} \\
=& L(X_t | \bar X_t)
\end{aligned}\]
所以\(Y_t\)不是\(X_t\)的格兰格原因。

考虑瞬时因果性。
\[\begin{aligned}
L(Y_t | \bar X_t, \bar Y_t, X_t)
=& X_{t} + 0 (\text{注意}\varepsilon_t\text{与}\{X_s, \forall s\}\text{不相关} \\
=& X_t \\
\sigma(Y_t | \bar X_t, \bar Y_t, X_t)
=& \text{Var}(\varepsilon) \\
=& 1 < 2 = \sigma(Y_t | \bar X_t, \bar Y_t)
\end{aligned}\]
所以\(X_t\)是\(Y_t\)的瞬时格兰格原因。

\[\begin{aligned}
[aaa]
\end{aligned}\]

\printbibliography

\end{document}
